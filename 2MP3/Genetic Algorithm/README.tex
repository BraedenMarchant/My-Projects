\documentclass[12pt]{article}
\usepackage{ragged2e} % load the package for justification
\usepackage{hyperref}
\usepackage[utf8]{inputenc}
\usepackage{pgfplots}
\usepackage{tikz}
\usetikzlibrary{fadings}
\usepackage{filecontents}
\usepackage{multirow}
\usepackage{amsmath}
\pgfplotsset{width=10cm,compat=1.17}
\setlength{\parskip}{0.75em} % Set the space between paragraphs
\usepackage{setspace}
\setstretch{1.2} % Adjust the value as per your preference
\usepackage[margin=2cm]{geometry} % Adjust the margin
\setlength{\parindent}{0pt} % Adjust the value for starting paragraph
\usetikzlibrary{arrows.meta}
\usepackage{mdframed}
\usepackage{float}

\usepackage{hyperref}

% to remove the hyperline rectangle
\hypersetup{
	colorlinks=true,
	linkcolor=black,
	urlcolor=blue
}

\usepackage{xcolor}
\usepackage{titlesec}
\usepackage{titletoc}
\usepackage{listings}
\usepackage{tcolorbox}
\usepackage{lipsum} % Example text package
\usepackage{fancyhdr} % Package for customizing headers and footers



% Define the orange color
\definecolor{myorange}{RGB}{255,65,0}
% Define a new color for "cherry" (dark red)
\definecolor{cherry}{RGB}{148,0,25}
\definecolor{codegreen}{rgb}{0,0.6,0}



%%%%%%%%%%%%%%%%%%%%%%%%%%%%%%%%%%%%%%%%%%%%%%%%%%%%%%%%%%%%%%%%%%%%%
% Apply the custom footer to all pages
\pagestyle{fancy}

% Redefine the header format
\fancyhead{}
\fancyhead[R]{\textcolor{orange!80!black}{\itshape\leftmark}}

\fancyhead[L]{\textcolor{black}{\thepage}}


% Redefine the footer format with a line before each footnote
\fancyfoot{}
\fancyfoot[C]{\footnotesize B. Marchant, McMaster University, Programming for Mechatronics - MECHTRON 2MP3. \footnoterule}

% Redefine the footnote rule
\renewcommand{\footnoterule}{\vspace*{-3pt}\noindent\rule{0.0\columnwidth}{0.4pt}\vspace*{2.6pt}}

% Set the header rule color to orange
\renewcommand{\headrule}{\color{orange!80!black}\hrule width\headwidth height\headrulewidth \vskip-\headrulewidth}

% Set the footer rule color to orange (optional)
\renewcommand{\footrule}{\color{black}\hrule width\headwidth height\headrulewidth \vskip-\headrulewidth}

%%%%%%%%%%%%%%%%%%%%%%%%%%%%%%%%%%%%%%%%%%%%%%%%%%%%%%%%%%%%%%%%%%%%%


% Set the color for the section headings
\titleformat{\section}
{\normalfont\Large\bfseries\color{orange!80!black}}{\thesection}{1em}{}

% Set the color for the subsection headings
\titleformat{\subsection}
{\normalfont\large\bfseries\color{orange!80!black}}{\thesubsection}{1em}{}

% Set the color for the subsubsection headings
\titleformat{\subsubsection}
{\normalfont\normalsize\bfseries\color{orange!80!black}}{\thesubsubsection}{1em}{}


%%%%%%%%%%%%%%%%%%%%%%%%%%%%%%%%%%%%%%%%%%%%%%%%%%%%%%%%%%%%%%%%%%%%%
% Set the color for the table of contents
\titlecontents{section}
[1.5em]{\color{orange!80!black}}
{\contentslabel{1.5em}}
{}{\titlerule*[0.5pc]{.}\contentspage}

% Set the color for the subsections in the table of contents
\titlecontents{subsection}
[3.8em]{\color{orange!80!black}}
{\contentslabel{2.3em}}
{}{\titlerule*[0.5pc]{.}\contentspage}

% Set the color for the subsubsections in the table of contents
\titlecontents{subsubsection}
[6em]{\color{orange!80!black}}
{\contentslabel{3em}}
{}{\titlerule*[0.5pc]{.}\contentspage}


%%%%%%%%%%%%%%%%%%%%%%%%%%%%%%%%%%%%%%%%%%%%%%%%%%%%%%%%%%%%%%%%%%%%%
% set a format for the codes inside a box with C format
\lstset{
	language=C,
	basicstyle=\ttfamily,
	backgroundcolor=\color{blue!5},
	keywordstyle=\color{blue},
	commentstyle=\color{codegreen},
	stringstyle=\color{red},
	showstringspaces=false,
	breaklines=true,
	frame=single,
	rulecolor=\color{lightgray!35}, % Set the color of the frame
	numbers=none,
	numberstyle=\tiny,
	numbersep=5pt,
	tabsize=1,
	morekeywords={include},
	alsoletter={\#},
	otherkeywords={\#}
}




%\input listings.tex



% Define a command for inline code snippets with a colored and rounded box
\newtcbox{\codebox}[1][gray]{on line, boxrule=0.2pt, colback=blue!5, colframe=#1, fontupper=\color{cherry}\ttfamily, arc=2pt, boxsep=0pt, left=2pt, right=2pt, top=3pt, bottom=2pt}




\tikzset{%
	every neuron/.style={
		circle,
		draw,
		minimum size=1cm
	},
	neuron missing/.style={
		draw=none, 
		scale=4,
		text height=0.333cm,
		execute at begin node=\color{black}$\vdots$
	},
}



%%%%%%%%%%%%%%%%%%%%%%%%%%%%%%%%%%%%%%%%%%%%%%%%%%%%%%%%%%%%%%%%%%%%%

% Define a new tcolorbox style with default options
\tcbset{
	myboxstyle/.style={
		colback=orange!10,
		colframe=orange!80!black,
	}
}

% Define a new tcolorbox style with default options to print the output with terminal style


\tcbset{
	myboxstyleTerminal/.style={
		colback=blue!5,
		frame empty, % Set frame to empty to remove the fram
	}
}

\mdfdefinestyle{myboxstyleTerminal1}{
	backgroundcolor=blue!5,
	hidealllines=true, % Remove all lines (frame)
	leftline=false,     % Add a left line
}


\begin{document}
	
	\justifying
	
	\begin{center}
		\textbf{{\large Minimizing The Ackley Function}}
		
		\textbf{Developing a Basic Genetic Optimization Algorithm in C} 
		
		Braeden Marchant
	\end{center}
	

	
	
	
	\section{Introduction}
	In this project I will use a Genetic Algorithm to minimize the Ackley function. To do this, my goal is to minimize my fitness while using various functions and algorithms. The most important function is the crossover function, which emulates parents having children and creating new populations. I am also trying to optimize the time it takes to run this code and lower the CPU time.
	
	\section{Makefile}

    The Makefile compiles each all of the required files for the main function to run.
    The compiler used is set to gcc and the compilation flags include -Wall, -Wextra, and -lm. The -W flags are to enable warning messages, the -lm is to link the math library for the math constants in OF.c. 
    There are several targets in my Makefile. The first is "all" which is what will run when I just run 'make' without specifying a target. It will build the executable 'GA' which depedns on three object files. GA.o, OF.o, and functions.o.
    These object files are responsible for compling the corresponding .c files.
    The clean target removes the GA executable and all the object files when running 'make clean'.

	\subsection{Results and Tables}
	
	I have ran my program using the parameters defined in the following tables:

    \begin{table}[h!]
		\caption{Results with Crossover Rate = 0.5 and Mutation Rate = 0.05}
		\label{table:1}
		\centering
		\begin{tabular}{c c c c c c}
			\hline
			Pop Size & Max Gen & \multicolumn{3}{c}{Best Solution} & CPU time (Sec) \\
			& & $x_1$ & $x_2$ & Fitness & \\
			\hline
            10  & 100    & -2.629627 & -4.586838 & 0.078207 & 0.000341 \\
			100 & 100    & -4.601986 & -4.860407 & 0.074511 & 0.000287 \\
			1000& 100    & 4.548867 & 4.572635 & 0.069985 & 0.220027 \\
			10000& 100    & 4.563190 & 4.593371 & 0.069943 & 20.719681 \\
			\hline
			1000  & 1000   & -4.602676 & 4.592629 & 0.069918 & 2.179273 \\
			1000 & 10000  & -4.728659 & -4.570522 & 0.070263 & 21.704112 \\
			1000& 100000 & -4.561901 & -4.592675 & 0.069945 & 216.331819 \\
			1000& 1000000 & 4.562983 & 5.480379 & 0.069926 & 1987.210923 \\ 
			\hline
		\end{tabular}
	\end{table}

	\begin{table}[h!]
		\caption{Results with Crossover Rate = 0.5 and Mutation Rate = 0.2}
		\label{table:1}
		\centering
		\begin{tabular}{c c c c c c}
			\hline
			Pop Size & Max Gen & \multicolumn{3}{c}{Best Solution} & CPU time (Sec) \\
			& & $x_1$ & $x_2$ & Fitness & \\
			\hline
			10  & 100    & -4.234242 & -4.793568 & 0.074511 & 0.000287 \\
			100 & 100   & -4.511490 & -4.557875 & 0.070113 & 0.004479 \\
			1000& 100    & 4.659980 & 4.600411 & 0.070002 & 0.354212 \\
			10000& 100    & 4.598250 & -4.601851 & 0.069917 & 33.816622 \\
			\hline
			1000  & 1000   & -4.614045 & 4.568012 & 0.069937 & 3.503184 \\
			1000 & 10000  & 4.618382 & 4.521774 & 0.070034 & 34.979671 \\
			1000& 100000 & -4.589894 & 4.587132 & 0.069921 & 259.063315 \\
			1000& 1000000 & -4.582750 & -4.724086 & 0.069912 & 2460.183864 \\
			\hline
		\end{tabular}
	\end{table}
	
		
\end{document}