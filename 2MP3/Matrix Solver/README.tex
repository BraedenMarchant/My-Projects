\documentclass[12pt]{article}
\usepackage{ragged2e} % load the package for justification
\usepackage{hyperref}
\usepackage[utf8]{inputenc}
\usepackage{pgfplots}
\usepackage{tikz}
\usetikzlibrary{fadings}
\usepackage{filecontents}
\usepackage{multirow}
\usepackage{amsmath}
\pgfplotsset{width=10cm,compat=1.17}
\setlength{\parskip}{0.75em} % Set the space between paragraphs
\usepackage{setspace}
\setstretch{1.2} % Adjust the value as per your preference
\usepackage[margin=2cm]{geometry} % Adjust the margin
\setlength{\parindent}{0pt} % Adjust the value for starting paragraph
\usetikzlibrary{arrows.meta}
\usepackage{mdframed}
\usepackage{float}

\usepackage{hyperref}

% to remove the hyperline rectangle
\hypersetup{
	colorlinks=true,
	linkcolor=black,
	urlcolor=blue
}

\usepackage{xcolor}
\usepackage{titlesec}
\usepackage{titletoc}
\usepackage{listings}
\usepackage{tcolorbox}
\usepackage{lipsum} % Example text package
\usepackage{fancyhdr} % Package for customizing headers and footers



% Define the orange color
\definecolor{myorange}{RGB}{255,65,0}
% Define a new color for "cherry" (dark red)
\definecolor{cherry}{RGB}{148,0,25}
\definecolor{codegreen}{rgb}{0,0.6,0}



%%%%%%%%%%%%%%%%%%%%%%%%%%%%%%%%%%%%%%%%%%%%%%%%%%%%%%%%%%%%%%%%%%%%%
% Apply the custom footer to all pages
\pagestyle{fancy}

% Redefine the header format
\fancyhead{}
\fancyhead[R]{\textcolor{orange!80!black}{\itshape\leftmark}}

\fancyhead[L]{\textcolor{black}{\thepage}}


% Redefine the footer format with a line before each footnote
\fancyfoot{}
\fancyfoot[C]{\footnotesize B. Marchant, McMaster University, Programming for Mechatronics - MECHTRON 2MP3. \footnoterule}

% Redefine the footnote rule
\renewcommand{\footnoterule}{\vspace*{-3pt}\noindent\rule{0.0\columnwidth}{0.4pt}\vspace*{2.6pt}}

% Set the header rule color to orange
\renewcommand{\headrule}{\color{orange!80!black}\hrule width\headwidth height\headrulewidth \vskip-\headrulewidth}

% Set the footer rule color to orange (optional)
\renewcommand{\footrule}{\color{black}\hrule width\headwidth height\headrulewidth \vskip-\headrulewidth}

%%%%%%%%%%%%%%%%%%%%%%%%%%%%%%%%%%%%%%%%%%%%%%%%%%%%%%%%%%%%%%%%%%%%%


% Set the color for the section headings
\titleformat{\section}
{\normalfont\Large\bfseries\color{orange!80!black}}{\thesection}{1em}{}

% Set the color for the subsection headings
\titleformat{\subsection}
{\normalfont\large\bfseries\color{orange!80!black}}{\thesubsection}{1em}{}

% Set the color for the subsubsection headings
\titleformat{\subsubsection}
{\normalfont\normalsize\bfseries\color{orange!80!black}}{\thesubsubsection}{1em}{}


%%%%%%%%%%%%%%%%%%%%%%%%%%%%%%%%%%%%%%%%%%%%%%%%%%%%%%%%%%%%%%%%%%%%%
% Set the color for the table of contents
\titlecontents{section}
[1.5em]{\color{orange!80!black}}
{\contentslabel{1.5em}}
{}{\titlerule*[0.5pc]{.}\contentspage}

% Set the color for the subsections in the table of contents
\titlecontents{subsection}
[3.8em]{\color{orange!80!black}}
{\contentslabel{2.3em}}
{}{\titlerule*[0.5pc]{.}\contentspage}

% Set the color for the subsubsections in the table of contents
\titlecontents{subsubsection}
[6em]{\color{orange!80!black}}
{\contentslabel{3em}}
{}{\titlerule*[0.5pc]{.}\contentspage}


%%%%%%%%%%%%%%%%%%%%%%%%%%%%%%%%%%%%%%%%%%%%%%%%%%%%%%%%%%%%%%%%%%%%%
% set a format for the codes inside a box with C format
\lstset{
	language=C,
	basicstyle=\ttfamily,
	backgroundcolor=\color{blue!5},
	keywordstyle=\color{blue},
	commentstyle=\color{codegreen},
	stringstyle=\color{red},
	showstringspaces=false,
	breaklines=true,
	frame=single,
	rulecolor=\color{lightgray!35}, % Set the color of the frame
	numbers=none,
	numberstyle=\tiny,
	numbersep=5pt,
	tabsize=1,
	morekeywords={include},
	alsoletter={\#},
	otherkeywords={\#}
}




%\input listings.tex



% Define a command for inline code snippets with a colored and rounded box
\newtcbox{\codebox}[1][gray]{on line, boxrule=0.2pt, colback=blue!5, colframe=#1, fontupper=\color{cherry}\ttfamily, arc=2pt, boxsep=0pt, left=2pt, right=2pt, top=3pt, bottom=2pt}




\tikzset{%
	every neuron/.style={
		circle,
		draw,
		minimum size=1cm
	},
	neuron missing/.style={
		draw=none, 
		scale=4,
		text height=0.333cm,
		execute at begin node=\color{black}$\vdots$
	},
}



%%%%%%%%%%%%%%%%%%%%%%%%%%%%%%%%%%%%%%%%%%%%%%%%%%%%%%%%%%%%%%%%%%%%%

% Define a new tcolorbox style with default options
\tcbset{
	myboxstyle/.style={
		colback=orange!10,
		colframe=orange!80!black,
	}
}

% Define a new tcolorbox style with default options to print the output with terminal style


\tcbset{
	myboxstyleTerminal/.style={
		colback=blue!5,
		frame empty, % Set frame to empty to remove the fram
	}
}

\mdfdefinestyle{myboxstyleTerminal1}{
	backgroundcolor=blue!5,
	hidealllines=true, % Remove all lines (frame)
	leftline=false,     % Add a left line
}


\begin{document}
	
	\justifying
	
	\begin{center}
		\textbf{{\large Solving System of Linear Equations}}
		
		\textbf{Developing a Jacobi Method solver in C} 
		
		Braeden Marchant
	\end{center}
	

	
	
	
	\section{Introduction}
	In this project I will solve a system of Linear equations starting with a .mtx file that is converted into CSR format to save memmory. 
	I will then use the Jacobi Method to solve the system in an iterative method. To then check the accuracy of the solution, the residual norm will be calculated.
	I am also trying to optimize the time it takes to run the solver and lower the CPU time.

	The Jacobi method is an iterative method to solve systems of Linear Equations.
	The main iteration step uses the formula:
	\begin{figure}[H]
	\centering
	\includegraphics{JacobiMethod.png}
	\end{figure}

	\section{What can it solve?}
	My solver is able to get results for the b1\_ss.mtx file, however the results are no correct as I get a residual of 15.338841. 
	This is because the b1ss.mtx file is not perfectly symmetric.
	I am able to solve ACTIVSg70K.mtx and 2cubes\_sphere.mtx but tmt\_sym.mtx and StocF-1465.mtx are too large to solve in a reasonable amount of time.
	
	\section{Makefile}

    The Makefile compiles each all of the required files for the main function to run.
    The compiler used is set to gcc and the compilation flags include -Wall, -Wextra, -lm and -pg. The -W flags are to enable warning messages, the -lm is to link the math library. 
    There are several targets in my Makefile. The first is "all" which is what will run when I just run 'make' without specifying a target. It will build the executable 'main.o and functions.o'.
    These object files are responsible for compling the corresponding .c files.
    The clean target removes the main executable and all the object files when running 'make clean'.

	\section{Vtune and gcov}

	Here is the summary when running vtune for the LFAT5.mtx file
	\begin{figure}[H]
		\caption{Vtune Summary}
		\centering
		\includegraphics{V_LFAT5.png}
	\end{figure}

	\begin{figure}[H]
		\centering
		\includegraphics{vtune.png}
	\end{figure}

	Here is an example of the gcov data for my main function when running the LFAT5.mtx file:
	\begin{mdframed}[style=myboxstyleTerminal1]
		\begin{verbatim}
			-:    0:Source:main.c
			-:    0:Graph:main.gcno
			-:    0:Data:main.gcda
			-:    0:Runs:1
			-:    1:#include <stdio.h>
			-:    2:#include <stdlib.h>
			-:    3:#include <time.h>
			-:    4:#include <math.h>
			-:    5:#include "functions.h"
			-:    6:
			1:    7:int main(int argc, char *argv[])
			-:    8:{
			-:    9:
			-:   10:    // <Handle the inputs here>
			1:   11:    const char *filename = argv[1];
			-:   12:
			-:   13:    CSRMatrix A;
			1:   14:    ReadMMtoCSR(filename, &A);
			-:   15:    
			-:   16:
			-:   17:    // Initializing all the vector b (in Ax=b)
			1:   18:    double *b = (double *)malloc(A.num_rows * sizeof(double));
			-:   19:    // Set all elements of b to 1
		   15:   20:    for (int i = 0; i < A.num_rows; ++i)
			-:   21:    {
		   14:   22:        b[i] = 1.0;
			-:   23:    }
			-:   24:
			-:   25:    // <The rest of your code goes here>
			1:   26:    double *y = (double *)malloc(A.num_rows * sizeof(double));
			1:   27:    double *x = (double *)malloc(A.num_rows * sizeof(double));
			1:   28:    double *r = (double *)malloc(A.num_rows * sizeof(double));
			-:   29:
			-:   30:    //DisplayCSR(&A);
			-:   31:    
			-:   32:    clock_t start_time, end_time;
			-:   33:    double cpu_time_used;
			1:   34:    start_time = clock();
			-:   35:
			1:   36:    jacobiSolver(&A, b, x);
			-:   37:
			1:   38:    end_time = clock();
			1:   39:    cpu_time_used = ((double)(end_time - start_time)) / CLOCKS_PER_SEC;
			-:   40:    //printf("CPU time: %f seconds\n", cpu_time_used);
			-:   41:
			-:   42:    //Prints out matrix X
			-:   43:    // printf("\nMatrix X:\n");
			-:   44:    // for (int i = 0; i < A.num_rows; ++i) {
			-:   45:    //     printf("x[%d]=%.6f\n",i,x[i]);
			-:   46:    // }
			-:   47: 
			1:   48:    spmv_csr(&A,x,y);
			-:   49:
			-:   50:    //prints out matrix Y
			-:   51:    /*printf("\nMatrix A*X:\n");
			-:   52:    for (int i=0; i < A.num_rows; ++i){
			-:   53:        printf("y[%d]=%.6f\n",i,y[i]);
			-:   54:    }*/
			-:   55:    
			1:   56:    compute_residual(&A, b, y, r);
			-:   57:        
			1:   58:    printf("The matrix name: %s\n",filename);
			1:   59:    printf("The dimension of the matrix: %d by %d\n",A.num_rows,A.num_rows);
			1:   60:    printf("Number of non-zeros: %d\n", A.num_non_zeros);
			1:   61:    printf("CPU time taken to solve Ax=b: %f seconds\n", cpu_time_used);
			1:   62:    compute_norm(&A, r);
			-:   63:    
			-:   64:}	
		\end{verbatim}
	\end{mdframed}

	Here is an example of the gcov data for my functions.c file when running the LFAT5.mtx file:
	\begin{mdframed}[style=myboxstyleTerminal1]
		\begin{verbatim}
			-:    0:Source:functions.c
			-:    0:Graph:functions.gcno
			-:    0:Data:functions.gcda
			-:    0:Runs:2
			-:    1:#include <stdbool.h>
			-:    2:#include <stdio.h>
			-:    3:#include <stdlib.h>
			-:    4:#include <string.h>
			-:    5:#include "functions.h"
			-:    6:#include <math.h>
			-:    7:#define BUFFER 300
			-:    8:
			-:    9:struct CSRMatrix {
			-:   10:    double *csr_data;   // Array of non-zero values
			-:   11:    int *col_ind;       // Array of column indices
			-:   12:    int *row_ptr;       // Array of row pointers
			-:   13:    int num_non_zeros;  // Number of non-zero elements
			-:   14:    int num_rows;       // Number of rows in matrix
			-:   15:    int num_cols;       // Number of columns in matrix    
			-:   16:};
			-:   17:
			-:   18:struct SolverMatrix {
			-:   19:    double *csr_data;   // Array of non-zero values
			-:   20:    int *col_ind;       // Array of column indices
			-:   21:    int *row_ptr;       // Array of row pointers
			-:   22:    int num_non_zeros;  // Number of non-zero elements
			-:   23:    int num_rows;       // Number of rows in matrix
			-:   24:    int num_cols;       // Number of columns in matrix    
			-:   25:};
			-:   26:
			-:   27:struct MarketMatrix {
			-:   28:    int row;
			-:   29:    int column;
			-:   30:    double value;
			-:   31:};
			-:   32:
			-:   33:
			1:   34:void sortMarketMatrix(int size, struct MarketMatrix mm[size], int totalRows){
			1:   35:    struct MarketMatrix *tempMM = malloc(size * sizeof(struct MarketMatrix));
			1:   36:    int index = 0;
			-:   37:
			-:   38:    //Sort mm by mm.row value and put into temp array
			-:   39:    //Start from first row (represented by i) and group struct values with the same row into a temp struct
			-:   40:    
		   15:   41:    for(int i = 1; i <= totalRows; i++){
		   14:   42:        int j = 0; 
			-:   43:
		  434:   44:        while(j < size){ 
		  420:   45:            if(mm[j].row == i){
		   30:   46:                tempMM[index] = mm[j];
		   30:   47:                index++;
			-:   48:            }
		  420:   49:            j++;
			-:   50:        }
			-:   51:    }  
			-:   52:
			-:   53:    //Replace unsorted market matrix struct with sorted temporary struct
		   31:   54:    for(int i = 0; i < size; i++){
		   30:   55:        mm[i] = tempMM[i];
			-:   56:    } 
			-:   57:
			1:   58:    free(tempMM); 
			-:   59:    
			1:   60:}
			-:   61:
			-:   62:
			2:   63:void ReadMMtoCSR(const char *filename, CSRMatrix *matrix){
			-:   64:    struct MarketMatrix *marketMatrix;
			-:   65:    
			2:   66:    FILE * file = fopen (filename, "r");
			2:   67:    if ( file == NULL ) {
			1:   68:        printf (" Failed to open file %s: \n", filename);
			1:   69:        return;
			-:   70:    } 
			-:   71:
			-:   72:    char line[BUFFER];
			1:   73:    int mmRowCount = 0;
			1:   74:    int mmCount = 0;
		   49:   75:    while (fgets(line, BUFFER, file) != NULL)
			-:   76:    {   
			-:   77:        //FOR DEBUG
			-:   78:        //printf("line: %s\n", line);
			-:   79:        
			-:   80:        //Ignore comment lines
		   48:   81:        if(line[0] != '%'){
			-:   82:            
		   31:   83:            mmRowCount++;
			-:   84:            
			-:   85:            //If first row, set applicable matrix info in CSR
		   31:   86:            if(mmRowCount == 1){
			1:   87:                sscanf(line, "%d %d %d", &matrix->num_rows, &matrix->num_cols, &matrix->num_non_zeros);
			-:   88:
			-:   89:                //printf("rows: %d, columns: %d, non-zeroes: %d", matrix->num_rows, matrix->num_cols, matrix->num_non_zeros);
			-:   90:
			-:   91:                //allocate memory for arrays
			1:   92:                matrix->csr_data = (double*)malloc(matrix->num_non_zeros * sizeof(double));
			1:   93:                matrix->col_ind = (int*)malloc(matrix->num_non_zeros * sizeof(int));
			1:   94:                matrix->row_ptr = (int*)malloc((matrix->num_rows + 1) * sizeof(int));
			-:   95:                
			1:   96:                marketMatrix = malloc(matrix->num_non_zeros * sizeof(struct MarketMatrix));
			-:   97:
			1:   98:                continue;
			-:   99:            }
			-:  100:                        
		   30:  101:            sscanf(line, "%d %d %lf", &marketMatrix[mmCount].row, &marketMatrix[mmCount].column, &marketMatrix[mmCount].value);
		   30:  102:            mmCount++;
			-:  103:        }
			-:  104:    }
			-:  105:
			-:  106:    // printf("\nUNSORTED: \n");
			-:  107:    // for(int i = 0; i < matrix->num_non_zeros; i++){
			-:  108:    //     printf("%d %d %lf\n", mm[i].row, mm[i].column, mm[i].value);
			-:  109:    // }
			-:  110:    
			-:  111:    //Sort Market Matrix in row ascending order to correctly retrieve values to set in CSR
			1:  112:    sortMarketMatrix(matrix->num_non_zeros, marketMatrix, matrix->num_rows); 
			-:  113:
			-:  114:    // printf("\nSORTED: \n");
			-:  115:    // for(int i = 0; i < matrix->num_non_zeros; i++){
			-:  116:        
			-:  117:    //     printf("%d %d %lf\n", mm[i].row, mm[i].column, mm[i].value);
			-:  118:    // }
			-:  119:
			1:  120:    int nzTotalInRow = 1; //counts the number of non-zero values in a row
			1:  121:    int nzRowIndex = 0; //
			1:  122:    int *nzRowSet = malloc(matrix->num_rows * sizeof(int)); //stores the total number of non-zero values per row
			-:  123:    
			-:  124:    //Set CSR values and do initial calculation for row pointer values
		   31:  125:    for(int i = 0; i < matrix->num_non_zeros; i++){
		   30:  126:        matrix->col_ind[i] = marketMatrix[i].column-1;
		   30:  127:        matrix->csr_data[i] = marketMatrix[i].value;
			-:  128:   
		   30:  129:        if(marketMatrix[i].row == marketMatrix[i+1].row){
		   16:  130:            nzTotalInRow++; //count number of non-zero values in row
			-:  131:        } else {
		   14:  132:            nzRowSet[nzRowIndex] = nzTotalInRow;
		   14:  133:            nzTotalInRow = 1;
			-:  134:
		   14:  135:            if(nzRowIndex < matrix->num_rows - 1){
		   13:  136:                nzRowIndex++;
			-:  137:            }
			-:  138:        }
			-:  139:    }
			-:  140:
			-:  141:    //Calculate row pointer values
		   16:  142:    for(int i = 0; i <= nzRowIndex + 1; i++){ 
			-:  143:        //printf("i: %d, nzRowSet: %d\n", i, nzRowSet[i]);    
		   15:  144:        if(i == 0){
			1:  145:            matrix->row_ptr[i] = 0;  
			-:  146:        } else {
		   14:  147:            matrix->row_ptr[i] = matrix->row_ptr[i-1] + nzRowSet[i-1];
			-:  148:        }
			-:  149:    }
			-:  150:
			-:  151:    //free memmory
			1:  152:    free(nzRowSet);
			1:  153:    free(marketMatrix);
			1:  154:    fclose(file);
			-:  155:}
			-:  156:
			-:  157:
			2:  158:void spmv_csr(const CSRMatrix *A, const double *x, double *y){
		   16:  159:    for (int i=0; i < A->num_rows; ++i){
		   14:  160:        y[i]=0.0;
			-:  161:        // goes through each row using the row ptr and multiplies the value of A by each of the x columns 
		   44:  162:        for (int j = A->row_ptr[i]; j < A->row_ptr[i+1]; j++){
		   30:  163:            y[i] += (A->csr_data[j] * x[A->col_ind[j]]);
			-:  164:        }
			-:  165:    }
			2:  166:}
			-:  167:
			-:  168:
		#####:  169:void DisplayCSR(const CSRMatrix *A){
			-:  170:    // printf("matrix num row: %d, matrix num col: %d, matrix num non zeros: %d\n", matrix->num_rows, matrix->num_cols, matrix->num_non_zeros);
		#####:  171:    printf("\nNumber of non-zeros: %d\n", A->num_non_zeros);
		#####:  172:    printf("Row Pointer: ");
		#####:  173:    for(int i = 0; i < (A->num_rows + 1); i++){
		#####:  174:        printf("%d ", A->row_ptr[i]);
			-:  175:    }
			-:  176:
		#####:  177:    printf("\nColumn Index: ");
		#####:  178:    for(int i = 0; i < A->num_non_zeros; i++){
		#####:  179:        printf("%d ", A->col_ind[i]);
			-:  180:    }
			-:  181:
		#####:  182:    printf("\nValues: ");
		#####:  183:    for(int i = 0; i < A->num_non_zeros; i++){
		#####:  184:        printf("%lf ", A->csr_data[i]);
			-:  185:    }
			-:  186:
		#####:  187:}
			-:  188:
			-:  189:
			-:  190:#define MAX_ITER 1000
			-:  191:#define EPSILON 1e-6
			-:  192:
			-:  193://Function to perform Jacobi iteration
			2:  194:void jacobiSolver(const CSRMatrix *A,const double *b, double *x) {
			2:  195:    int n=A->num_rows;
			2:  196:    double *x_new = (double *)malloc(n * sizeof(double));
			-:  197:    int k;
			-:  198:
			7:  199:    for (k = 0; k < MAX_ITER; ++k) {
		   91:  200:        for (int i = 0; i < n; ++i) {
		   84:  201:            double sigma = 0.0;
		  264:  202:            for (int j = A->row_ptr[i]; j < A->row_ptr[i + 1]; ++j) {
		  180:  203:                if (A->col_ind[j] != i) {
		   96:  204:                    sigma += A->csr_data[j] * x[A->col_ind[j]];
			-:  205:                }
			-:  206:            }
		   84:  207:            x_new[i] = (b[i] - sigma) / A->csr_data[A->row_ptr[i + 1] - 1];
			-:  208:        }
			-:  209:
			-:  210:        // Check for convergence
			7:  211:        double max_diff = 0.0;
		   91:  212:        for (int i = 0; i < n; ++i) {
		   84:  213:            double diff = fabs(x_new[i] - x[i]);
		   84:  214:            if (diff > max_diff) {
		   23:  215:                max_diff = diff;
			-:  216:            }
			-:  217:        }
			-:  218:
			-:  219:        //prints num iterations after convergence is reached
			7:  220:        if (max_diff < EPSILON) {
			2:  221:            printf("Convergence reached after %d iterations.\n", k + 1);
			2:  222:            break;
			-:  223:        }
			-:  224:
			-:  225:        // Update x for the next iteration
		   75:  226:        for (int i = 0; i < n; ++i) {
		   70:  227:            x[i] = x_new[i];
			-:  228:        }
			-:  229:    }
			2:  230:    free(x_new);
			2:  231:}
			-:  232:
			2:  233:void compute_residual(const CSRMatrix *A, double *b, double *y, double *r){
		   16:  234:    for (int i = 0; i < A->num_rows; ++i){
		   14:  235:        r[i]=y[i]-b[i];
			-:  236:        //printf("r[%d]=%f\n",i,r[i]);
			-:  237:    }
			2:  238:}
			-:  239:
			2:  240:void compute_norm(const CSRMatrix *A, double *r){
			2:  241:    double mag = 0.0;
			2:  242:    double sum = 0.0;
		   16:  243:    for (int i = 0; i < A->num_rows; ++i){
		   14:  244:        sum += r[i]*r[i];
			-:  245:        //printf("sum:%f",sum);
		   14:  246:        mag = sqrt(sum);
			-:  247:    }
			2:  248:    printf("Residual Norm: %f\n", mag);
			2:  249:}		
		\end{verbatim}
	\end{mdframed}
	We can see here which lines of code are being run and the number of times they are run.
	It is mostly the loops inside my sort function and the jacobi solve function that take the most amount of time
	
	\subsection{Results and Tables}
	
	I have run my program for the following .mtx files and these are the results

    \begin{table}[h!]
		\caption{Results}
		\label{table:1}
		\centering
		\begin{tabular}{c c c c c c}
			\hline
			Problem & \multicolumn{2}{c}{Size} & non-zeroes & CPU time (Sec) & Residual \\
			& & $row$ & $column$ & \\
			\hline
            LFAT5.mtx & 14 & 14 & 30 & 0.000000 & 0.000000 \\
			LF10.mtx & 18 &  & 18 & 0.000000 & 0.000000 \\
			ex3.mtx & 1821 & 1821 & 27253 & 0.030013 & 0.014618 \\
			jnlbrng1.mtx & 40000 & 40000 & 119600 & 0.010001 & 0.000560 \\
			ACTIVSg70K.mtx & 69999   & 69999 & 154313 & 0.020005 & 0.000122 \\
			2cubes\_sphere.mtx & 101492  & 101492 & 874378 & 0.050029 & 0.002300 \\
			tmt\_sym.mtx & n/a & n/a & n/a & n/a & n/a \\
			StocF-1465.mtx & n/a & n/a & n/a & n/a & n/a \\ 
			\hline
		\end{tabular}
	\end{table}

	\begin{figure}[H]
		\caption{LFAT5.mtx}
		\centering
		\includegraphics{LFAT5.png}
	\end{figure}


	\begin{figure}[H]
		\caption{LF10.mtx}
		\centering
		\includegraphics{LF10.png}
	\end{figure}


	\begin{figure}[H]
		\caption{ex3.mtx}
		\centering
		\includegraphics{ex3.png}
	\end{figure}

	\begin{figure}[H]
		\caption{jnlbrng1.mtx}
		\centering
		\includegraphics{jnlbrng1.png}
	\end{figure}


	\begin{figure}[H]
		\caption{ACTIVSg70K.mtx}
		\centering
		\includegraphics{ACTIVSg70K.png}
	\end{figure}

	\begin{figure}[H]
		\caption{2cubes\_sphere.mtx}
		\centering
		\includegraphics{2cubes_sphere.png}
	\end{figure}
	
	\begin{figure}[H]
		\caption{b1\_ss.mtx}
		\centering
		\includegraphics{b1_ss.png}
	\end{figure}

	
		
\end{document}
